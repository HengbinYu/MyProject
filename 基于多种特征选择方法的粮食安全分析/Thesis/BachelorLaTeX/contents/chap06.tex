% 本文件是示例论文的一部分
% 论文的主文件位于上级目录的 `main.tex`

\chapter{总结与展望}
本文选择我国受俄乌冲突影响较大的粮食作物—玉米作为研究对象,以粗糙集属性约简方法作为基础,选择了基于属性重要度的模糊粗糙集属性约简算法作为核心算法,建立了分析我国玉米产量影响因素的粮食生产模型,对各省粮食生产数据进行了特征选择,选择出了各省最重要的10个因素,并采用预测模型对特征选择结果进行数值实验分析,以便评价特征选择结果的优劣。最后对特征选择结果进行了可视化结果分析,对我国粮食安全做出了科学建议。

在特征选择部分,本文选择基于属性重要度的模糊粗糙集属性约简算法(FRAR)、基于 Lasso 的特征选择算法与基于随机森林变量重要性的特征选择算法(RF)对各省粮食生产数据进行特征选择,分别选择出各省影响玉米产量的最重要10个因素,发现FRAR方法在耗时上仅为Lasso方法的一半,略慢于RF方法,并通过预测模型数值实验分析特征选择结果的优劣。

在预测模型数值实验中,本文通过比较经由三种不同特征选择方法处理得到的不同约简数据集在预测模型上的表现,分析不同特征选择方法得到的特征选择结果的优劣。得到FRAR约简数据集在训练耗时、训练误差、测试误差和预测精度方面较Lasso约简数据集有着很大的优势,且在表现上非常接近表现最佳的RF约简数据集,说明在本文实验中模糊粗糙集属性约简算法在玉米生产数据的特征选择上有着较大的优势。

对于数值实验结果,本文选择了随机森林回归模型作为预测模型,而特征选择方法其中之一就是随机森林变量重要性方法,该特征选择方法作为随机森林模型的一部分,理论上其特征选择结果应对随机森林预测模型有着较好的拟合度,因而可以将其约简数据集在预测模型中的表现作为一个较优的参考。而FRAR方法在实验结果各方面表现上均非常接近于表现最优的RF方法,进而侧面说明了FRAR方法在特征选择方面的优势。

在结果分析中,本文对FRAR方法的特征选择结果进行了可视化结果分析,分别从化肥投入情况、成本投入情况、排灌建设情况及机械化建设情况四个方面对影响玉米产量的影响因素进行分析,得到了影响各省玉米生产的重要因素,并对我国的粮食安全提供了科学的政策建议。

在未来的工作中,可以再利用聚类算法对不同省份的结果进行聚类,以便找到不同省份间玉米生产的共通性。同时还可以将粗糙集方法应用到其他粮食的产量影响因素的分析中,在特征种类上引入更丰富的解释变量,充分发挥粗糙集方法在处理不确定性数据上的优势,并根据不同的数据集的数据类型选择针对性更强的粗糙集算法,以便得到更准确的特征选择结果,进而对粮食安全进行更准确的分析。
%%% Local Variables: 
%%% mode: latex
%%% TeX-master: "../main.tex"
%%% End:
