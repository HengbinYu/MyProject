% 本文件是示例论文的一部分
% 论文的主文件是位于上级目录的 `main.tex`

\chapter{引言}
本章主要介绍粗糙集方法及我国粮食安全现状,粗糙集方法分析粮食安全的目的及意义。

\section{研究背景}
俄乌冲突作为2022年世界经济形势发展的最大影响因素,对全球各国经济发展、能源供应、资源贸易等方面都造成了较为严重的冲击,其中对全球粮食等农产品产业链产生了重要影响。

俄罗斯与乌克兰作为国际上重要的粮食生产与出口大国,在全球粮食产业链中具有着重要的地位。在冲突爆发前的2021年中,两国大麦、小麦和玉米的合计产量占据全球总量的19\%、14.3\%和4.6\%,且全球32.5\%的大麦、30.6\%的小麦及19.7\%的玉米出口均来自俄乌两国\cite{俄乌冲突的蝴蝶效应与中国粮食安全的地缘风险}。又因为俄乌两国自身在粮食消费方面的数量有限,其粮食出口量在全球中占据的份额远大于产量所占据的份额,因此两国在全球的粮食贸易中占有重要的地位。同时,俄乌两国作为全球的粮食出口大国,其粮食主要出口至北非、中东、南亚地区的低收入缺粮国,在维持这些国家的粮食供给方面起着重要作用。根据联合国粮食及农业组织数据库的统计显示,一些非洲、亚洲等地一些最不发达国家或低收入国家对俄乌粮食有着很强的依赖度,俄乌两国的粮食出口变化将会对很多国家造成较为严重的影响\cite{乌克兰危机对全球粮食安全的影响与中国应对策略}。此外,俄乌不仅是主要的粮食生产与出口国,还是农作物肥料的主要供应国,两国在全球化肥出口贸易上占据全球总额的近四分之一,是全球化肥供应的主要力量\cite{俄乌冲突对全球和我国粮食安全短期影响及建议}。

俄乌冲突导致了全球农业产业链受到了严重的冲击,其造成了交战地区的粮食生产中断,粮食贸易、化肥和能源等资源的供应也几乎中断,同时北约对俄罗斯的经济制裁以及由此引发的反制裁措施严重破坏了区域经济秩序,进一步全面推高了农业生产成本。此外,粮食、化肥和能源的供应短缺预期也加剧了市场的恐慌,引发了农产品价格的普涨,严重影响了全球部分进口消费区域的粮食安全\cite{俄乌冲突对中国粮食安全的影响}。

俄乌冲突造成的全球粮食安全危机对我国有着一定的影响\cite{俄乌冲突对中国农产品供需直接影响分析,俄乌冲突对中国粮食安全的影响}。目前我国粮食安全总体形势良好,国内粮食供应充足,价格基本稳定,生产能够保障有序开展。从短期来看,对我国有着农产品价格有着一定的冲击,从长期来看,也为我国也为我国敲响了粮食安全的警钟,再次提醒我们落实粮食安全战略的艰巨性和复杂性。其中,乌克兰是全球第四大玉米出口国,2020年占全球玉米出口量的14.85\%,我国自乌克兰进口的玉米量占其出口总量的28.31\%。同时2021年我国自乌克兰进口的玉米量约占我国总玉米进口量的30\%,此次冲突对我国玉米供应将造成一定的影响,或部分推升国内玉米价格。而我国大麦进口主要用途为饲料用粮,作为玉米的主要替代品,乌克兰是中国大麦的主要供应国,乌克兰大麦占中国进口量的26\%,因此或也将部分推升国内大麦价格。我国小麦自给率高,国家小麦储备充足,市场稳定性较强,俄乌冲突对小麦影响有限。此外,我国化肥产业自给率较高,但国际化肥市场的价格飞涨,一定程度上会提高农民的农业生产成本。这对于我国外贸依存度最大的大豆以及相关饲料产业和养殖业有一定影响\cite{俄乌冲突对我国粮食安全的影响及其对策研究}。
\section{研究现状}

\subsection{粗糙集}
粗糙集的概念最初由Pawlak\cite{pawlak1995rough}提出,作为一种处理数据分析中的不精确、模糊和不确定性的数学方法。这一理论已被充分证明在成功解决各种问题方面具有实用性和通用性\cite{1998Rough,huang1992intelligent,ziarko1994rough}。粗糙集理论的一个重要应用是数据库中的属性约简。给定一个具有离散化属性值的数据集,可以找到原始属性的子集,该子集包含与原始属性相同的信息。区别于其他理论而言,属性约简的概念可以被视为粗糙集理论中最有力、最重要的结果。属性约简的粗糙集方法可以作为一种纯粹的结构方法,使用数据集中包含的信息降维并保留特征的含义\cite{tsang2008attributes}。

其中,基于属性重要度的属性约简算法作为粗糙集属性约简算法的一个分支,在计算属性重要度与属性约简方面有着较大的潜力。路松峰等人\cite{路松峰2008}对属性依赖没有考虑条件属性间的依赖关系的问题,提出了ARA属性约简算法;翟俊海等人\cite{翟俊海2014}针对属性依赖度的计算忽略了边界概率分布信息这一问题,提出了利用最小相关性和最大依赖度准则求决策表属性约简的改进算法。廖倩\cite{基于粗糙集和模糊粗糙集的属性约简研究}针对现有的基于正域的属性约简算法的弊端,提出了新的属性重要度计算方法,并针对模糊粗糙集属性约简算法中属性组合爆炸问题提出改进属性约简算法,有效减少了计算量。此外,还有其他研究着眼于属性重要度的度量与定义,针对不同问题提出了不同的算法。

然而,如\cite{jensen2004fuzzy}中所述,属性的值可以是符号值和实值。传统的粗糙集理论在处理此类实值属性时会遇到困难。解决这个问题的一种方法是预先离散属性值\cite{1998Discretization,},并创建一个新的具有符号属性值的数据集。然而,使用这种方法会造成信息丢失。另一种方法是使用模糊粗糙集。模糊粗糙集包含了模糊性和不可分辨性这两个相关但不同的概念,这两个概念都是由于知识中存在的不确定性而产生的。在模糊粗糙集中,模糊相似关系被用来表征两个对象之间的相似程度,而不是在清晰粗糙集中使用的等价关系。如果相似度为1,那么它们是不可分辨的。如果相似度为0,则它们是可辨别的。如果相似度取0到1之间的值,那么这两个对象在一定程度上相似。然而,不同的模糊相似性关系可能产生不同程度的相似性,因此信息丢失也是可能的。不合理的模糊相似关系会带来较大的信息损失。这个问题可以通过定义一个合理的模糊相似关系来解决。

模糊粗糙集(以下简记FRS)由Dubois和Prade\cite{dubois1990rough}于1990年提出,是一种将模糊集与粗糙集理论相结合的模型。模糊粗糙集理论在处理不确定性、不准确数据和不完整数据方面发挥着关键作用。1992年,Kuncheva\cite{KUNCHEVA1992147}首次提出了一种使用模糊粗糙集的特征选择方法。其利用弱模糊划分方法定义了模糊的正、负和边界区域,提出了特征选择的思想。然而,由于FRS在当时刚刚诞生,它们并没有引起足够的关注。2004年,Jensen等人\cite{2004Fuzzy}提出了一种基于FRS的属性约简方法。从那时起,基于FRS的属性约简受到了广泛的关注。根据所使用的不同约简规则,基于FRS的属性约简方法可以大致可分为三类:基于模糊依赖性的、基于模糊不确定性测度的以及基于模糊辨别矩阵的。

在此,重点介绍基于模糊依赖度的模糊粗糙集属性约简方法。2005年,由于\cite{2004Fuzzy}中提出的方法无法经常找到最优子集,Jenshen和Shen\cite{2005Fuzzy}提出了一种基于蚁群优化的特征选择策略来解决这个问题。然后将该方法应用于FRS下的最优特征子集的求解问题。针对\cite{2004Fuzzy}中算法的不收敛问题,在通过实验发现该问题后,Bhatt等人\cite{2005On,2005compact}提出了一种在域附近的改进策略,以提高算法的性能。考虑到在许多情况下可分辨性的概念更为自然有效,Cornelis等人\cite{2010Feature}提出了使用基于模糊容差关系的经典粗糙集框架的推广,然后引入了基于正域的模糊决策约简的概念。2010年,Hu等人\cite{2010Gaussian}将高斯核与FRS相结合,提出了一种基于高斯核的FRS模型,并基于所提出的模型设计了一种基于高斯核的特征选择算法。2013年,Mac ParthaláIn等人\cite{mac2013unsupervised}提出了一种基于模糊依赖性的无监督特征方法。在后续的研究中,大多数针对FRS方法耗时难以忍受的问题进行了加速研究,Qian等人\cite{qian2015fuzzy}提出了一种用于特征选择的加速器方法,Zhang等人\cite{zhang2018fuzzy}针对FRS代表实例的特征选择问题提出了一种基于代表实例的特性选择方法,Ni等人\cite{ni2019positive}基于Dubois和Prade的模糊粗糙模型,设计了一种基于正区域的加速器算法。这些研究都将模糊粗糙集方法推向了分析大规模和高维数据集的应用。

% 然而,FRS方法的时间消耗对于分析大规模和高维数据集来说是非常难以忍受的,在此之前虽然已经开发了许多基于启发式模糊粗糙集的特征选择算法,但这些方法通常在计算上仍然很耗时。2015年,为了进一步解决这个问题,Qian等人提出了一种用于特征选择的加速器方法[42]。他们选择了Hu等人[68]提出的模糊粗糙模型。Zhang等人在2018年研究了基于FRS的代表实例的特征选择问题,并提出了一种基于代表实例的特性选择方法[76]。2019年,Ni等人基于Dubois和Prade的模糊粗糙模型,设计了一种基于正区域的加速器算法[81]。此外,在模糊正区域和关键实例集的增量机制的基础上,他们提出了一种基于FRS的增量特征选择算法[82]。在上述基于依赖性的特征选择方法中,大多数方法都采用模糊关系的交集(连接)运算来构造属性约简的隶属度函数。采用不同的模糊上下近似算子也会有着不同的特征选择效果,针对不同类型的数据,在应用中需要根据数据结构选择合适的属性约简算法。

同时,模糊粗糙集已经在许多领域得到了应用。Xie等人\cite{xie2017evaluation}将模糊粗糙集应用于变压器油纸绝缘状态评估,为变压器油纸的绝缘状态评估提供了新的思路,在工程应用中具有实用价值。Zhang等人\cite{ZHANGChao1531}将模糊粗糙集应用于职业评估等多属性决策问题,并利用模糊集和粗糙集在不确定决策中的优势,为决策者提供有价值的决策模型。Guo等人\cite{rongchao2017fuzzy}在多标签分类任务中使用了模糊粗糙集模型,通过特征选择得到分类效果良好。曹愈远等人\cite{曹愈远2017基于模糊粗糙集}将模糊粗糙集与支持向量机相结合,用于航空发动机故障诊断,得到该方法具有较强的诊断能力,在不影响诊断率的情况下大大缩短了计算时间。总之,模糊粗糙集在数据挖掘中的应用是广泛的,尤其是在处理不确定、不完全和模糊信息问题时\cite{duan2020energy}。因此,本文考虑将模糊粗糙集应用到粮食安全影响因素的分析及预测中。

\subsection{粮食安全}

近年来,有部分研究着眼于利用经典变量选择方法对粮食安全进行分析。2015年,李岩岩等人\cite{基于SIR方法分析重庆市粮食产量}利用切片逆回归(SIR)方法对1978-2013年重庆市粮食总产量数据进行降维得到主要影响因素,再根据降维后的数据进行最小二乘拟合,得到了较好的预测效果。2016年,Jeong等人\cite{jeong2016random}评估了随机森林(RF)与多元线性回归(MLR)在预测小麦、玉米和土豆在全球和区域尺度上对气候和生物物理变量的作物产量反应,得到RF具有很高的预测作物产量的能力,并且在所有比较的性能统计中都优于MLR基准。2017年,马云倩等人\cite{孙君茂2018}利用Lasso方法筛选出影响我国粮食总产量的6个显著影响因素,将这6个因素作为输入因子构建粮食产量预测模型GM(1,6)进行预测,同时与原模型进行比较,预测出未来3年内我国粮食产量处于稳步增长的状态。2019年,Khaki等人\cite{khaki2019crop}设计了一种深度神经网络(DNN)方法,利用玉米杂交种的基因型和产量表现预测2017年的产量,并基于训练的DNN模型进行了特征选择,成功地降低输入空间的维数的同时又保证了预测精度没有显著下降,结果显示该模型显著优于Lasso、浅层神经网络和回归树等其他流行的方法。2022年,钟奇\cite{基于主成分分析的BP神经网络粮食产量预测模型}采用主成分分析法与BP神经网络相结合的预测模型,先利用主成分分析法对高维数据进行降维,再利用BP神经网络进行预测,得到了影响粮食生产的2个主要因素。

粗糙集方法在变量选择方面相较于传统方法有着很好的优势,仅有少数研究将粗糙集理论与粮食安全的分析结合起来。 2005年,Zhang与He\cite{2005Study}提出的基于粗糙集的灰色关联BP神经网络综合了粗糙集和灰色系统的理论,先利用灰色关联与粗糙集理论对1990-2002年我国粮食产量数据的条件属性进行约简得到关键因子,然后利用关键因子对BP神经网络进行训练和预测,大大提高了相比于一般神经网络的预测精度。2008年,韩万渠等人\cite{韩万渠2008基于粗糙集理论的我国粮食生产能力制约因素分析}基于粗糙集理论对我国粮食生产能力制约因素进行了分析,得到科研、化肥、机械是提高粮食产量的3个最重要的因素。2013年,欧阳浩等人\cite{基于粗糙集方法的广东省粮食产量影响因素分析}利用粗糙集方法对广东省粮食产量影响因素分析,得到了化肥用量、水库总容量、人均经营耕地面积三个关键影响因子。2015年,王丹丹等人\cite{基于粗糙集理论的河南省粮食产量预测研究}利用运用粗糙集理论对河南省粮食产量决策表进行属性约简。结果表明,粮食作物播种面积、化肥施用折纯量和农用机械总动力是河南省粮食产量的主要影响因素,且相对于粮食产量的重要性程度依次降低。

\section{研究的意义及创新点}
虽然我国粮食安全总体形势良好,但玉米、大豆、大麦等粮食进口需求相比较大的农产品价格仍然受到了俄乌冲突的影响。在俄乌冲突造成的粮食市场供应短缺下,很多国家将提高粮食作物的生产能力作为重要目标,扩大粮食作物的播种面积,这为我国敲响了粮食安全的警钟。

粮食安全的基石是确保粮食充足供给,粮食生产作为粮食安全中重要的一环,确保主粮作物的充足供应是粮食安全战略的关键。在当今地缘冲突影响全球粮食贸易的背景下,对我国影响较大的农产品的粮食生产将更值得我们重视。

粮食生产作为农业生产系统的一部分,其产量会受到主观客观等多种因素的影响,因此研究影响粮食生产的主要影响因素将对俄乌冲突背景下确保主粮作物的充足供应和产业调整具有重要意义。同时,为了保障粮食的安全,需要预知未来一段时间内粮食产量的变化情况,研究粮食产量的预测也是非常必要的。

基于粗糙集理论的方法与统计方法处理不确定问题完全不同,它不采用概率方法描述数据的不确定性,与这一领域传统的模糊集合论处理不精确数据的方法也不同。粗糙集方法能够分析隐藏在数据中的信息,而不需要关于数据的任何附加信息。该方法可以在保持属性集分类能力不变的情况下,进行属性约简,尽量删除冗余属性,留下对分类最重要的属性。

近5年来,绝大多数研究都倾向于利用机器学习模型来分析和预测粮食安全,近十年内很少能找到有利用粗糙集理论对粮食安全进行分析的研究,粗糙集理论的优势在于可以在尽量不损失属性集信息的条件下对数据进行降维约简,模糊粗糙集更是可以克服粗糙集方法对于离散化数据要求的局限性,而机器学习方法的一大难度在于训练过程的优化难题,将二者的优势加以结合的研究便显得尤为重要。

因此,本文选择我国受到俄乌冲突影响较严重代表性农作物-玉米作为研究对象,通过采用粗糙集方法分析影响其产量的主要因素,同时与传统特征选择方法进行比较,再利用预测模型对特征选择结果进行评价,进而得到粗糙集方法在特征选择方面相对于一般方法的优劣,最后根据特征选择结果对我国玉米生产及粮食安全提供建议。
\section{本文组织结构}
\begin{itemize}
    \item \ 第2章主要介绍基于粗糙集的属性约简,包括粗糙集的理论知识及属性约简算法。
    \item \ 第3章主要建立了基于粗糙集方法的粮食生产模型,介绍了研究数据、模型框架及模型方法。
    \item \ 第4章分别用三种特征选择方法对数据集进行处理,选出了每个特征选择方法下影响玉米生产因素最重要的前10个变量,同时得到了相应的约简数据集,并比较了三种不同算法的特征选择耗时。
    \item \ 第5章主要对第三章中的特征选择结果进行了数值实验验证。通过将不同方法下的约简数据集放入随机森林预测模型中进行训练与预测,比较每个省份的不同约简数据集在训练耗时、均方误差及预测精度上的差异,进而评价不同特征选择方法的特征选择效果。
    \item \ 第6章主要进行了粮食安全结果分析,通过对模糊粗糙集属性约简算法得到的各省影响玉米产量最重要的因素进行结果可视化,分别从化肥投入情况、成本投入情况、排灌建设情况及机械化建设情况四个方面对影响玉米产量的影响因素进行分析,对我国玉米生产及粮食安全提供了建议。
    \item \ 第7章主要总结了本文的工作,并对进一步的研究进行了展望。
\end{itemize}
