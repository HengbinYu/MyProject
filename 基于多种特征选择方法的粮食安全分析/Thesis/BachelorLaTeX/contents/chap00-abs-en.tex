% 英文摘要
The 2022 conflict between Russia and Ukraine, two major grain-producing countries, had a significant impact on global food security. China, as a major importer of corn from Ukraine, experienced disruptions in its corn and other grain supplies. This study aims to enhance China's food security in the context of the geopolitical conflict by focusing on corn, which was heavily affected by the conflict. Utilizing feature selection methods, the research analyzes corn production data from various provinces and identifies the ten most influential factors out of 26 explanatory variables affecting corn yield. The results are thoroughly analyzed and validated.

The core research method employed in this study is the fuzzy-rough set attribute reduction algorithm. A grain production model is developed to analyze the factors affecting corn yield. A comparison is made between the Lasso-based feature selection algorithm, the random forest variable importance-based feature selection algorithm (RF), and the fuzzy-rough set attribute reduction algorithm based on attribute importance (FRAR). The differences in feature selection performance among these algorithms are evaluated. It is found that the FRAR method takes only half the time compared to the Lasso method, albeit slightly slower than the RF method. Furthermore, numerical experiments using a random forest prediction model analyze and evaluate the reduced datasets obtained from different feature selection methods. The results demonstrate that the FRAR-reduced dataset outperforms the Lasso-reduced dataset in terms of training time, training set error, test set error, and prediction accuracy. Additionally, the FRAR-reduced dataset closely approaches the performance of the RF-reduced dataset, which exhibits the best performance across all aspects. Finally, the visual analysis of the feature selection results obtained from the FRAR method focuses on fertilizer input, cost input, irrigation and drainage construction, and mechanization construction. Conclusions regarding the factors influencing corn yield are drawn, and scientifically informed recommendations for China's food security policies are provided.