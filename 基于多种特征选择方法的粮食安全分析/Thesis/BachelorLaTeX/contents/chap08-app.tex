% 本文件是示例论文的一部分
% 论文的主文件位于上级目录的 `main.tex`
\chapter{特征选择结果及数据与代码}

\section{特征选择结果}
这里展示Lasso与随机森林变量重要性方法的特征选择排名前10的特征。
\begin{table}[!htbp]
    \centering
    \caption{Lasso:各省份最重要特征:1-5}
    \label{table:Feature5_Lasso1}
    \resizebox{\textwidth}{!}
    {
    \csvautobooktabular{data/ReducedFeatures_Lasso.csv}
    }
\end{table}
\setlength{\floatsep}{0pt}
\begin{table}[!htbp]
    \centering
    \caption{Lasso:各省份最重要特征:6-10}
    \label{table:Feature5_Lasso2}
    \resizebox{\textwidth}{!}
    {
    \csvautobooktabular{data/ReducedFeatures_Lasso2.csv}
    }
\end{table}
\setlength{\floatsep}{0pt}
\begin{table}[!htbp]
    \centering
    \caption{RF:各省份最重要特征:1-5}
    \label{table:Feature5_RF1}
    \resizebox{\textwidth}{!}
    {
    \csvautobooktabular{data/ReducedFeatures_RF.csv}
    }
\end{table}
\setlength{\floatsep}{0pt}
\begin{table}[!htbp]
    \centering
    \caption{RF:各省份最重要特征:6-10}
    \label{table:Feature5_RF2}
    \resizebox{\textwidth}{!}
    {
    \csvautobooktabular{data/ReducedFeatures_RF2.csv}
    }
\end{table}

\section{数据与代码}

由于本文实验数据及代码较多,不宜展示于附录中,故将数据与代码上传至以下网址:

https://github.com/HengbinYu/FoodSecurityAnalysisBasedonRoughSetMethod.git

% 先说结论:{\large\enquote{知网完全支持pdf查重}},学校学院也接收pdf格式的论文,这个无需担心。

% 如果导师只接受Word版论文,那也就没有办法了,你就用Word吧,只要下点功夫,也不是个事。建议大家提前和指导老师进行沟通,以确认能不能提交pdf格式论文。

% \section{批注}
% 在论文撰写过程中,pdf格式的论文,批注是一个问题,如果对\LaTeX 和基于Git的版本管理并不了解,就只能使用Adobe Acrobat、平板手写等软件,对pdf文件本身进行批注,相比于word确实有些麻烦。

% 强烈推荐使用Git\footnote{\url{https://git-scm.com/}}、Beyond Compare\footnote{\url{https://www.scootersoftware.com/}}等工具,辅以\LaTeX 本身的注释进行批注以及版本管理,非常清晰直观,操作也简单。

% \section{毕业设计与毕业论文的区别}
% 这里特别对使用本模板的本科同学们做出提醒,请查看毕业设计基本信息中的毕设类别,共有两类:\enquote{毕业设计}和\enquote{毕业论文}。因此在\verb!\documentclass[]{nwafuthesis}!的选项中需要标明\textbf{Design}(毕业设计)或者\textbf{Paper}(毕业论文),使论文使用正确的封面和独创性声明。

% \section{单面打印\& 双面打印}
% 学校并没有规定论文打印的方式,考虑到部分同学有双面打印的需求,可以在文档选项中使用oneside/twoside来切换单面打印和双面打印。

% \section{封面打印\& 装订}
% 建议大家去指定打印部门打印封面并装订,以免打印装订不合格。

% \section{批注}
% 在论文撰写过程中,pdf格式的论文,批注是一个问题,如果对\LaTeX 和基于Git的版本管理并不了解,就只能使用Adobe Acrobat、平板手写等软件,对pdf文件本身进行批注,相比于word确实有些麻烦。

% 强烈推荐使用Git\footnote{\url{https://git-scm.com/}}、Beyond Compare\footnote{\url{https://www.scootersoftware.com/}}等工具,辅以\LaTeX 本身的注释进行批注以及版本管理,非常清晰直观,操作也简单。

% \section{毕业设计与毕业论文的区别}
% 这里特别对使用本模板的本科同学们做出提醒,请查看毕业设计基本信息中的毕设类别,共有两类:\enquote{毕业设计}和\enquote{毕业论文}。因此在\verb!\documentclass[]{nwafuthesis}!的选项中需要标明\textbf{Design}(毕业设计)或者\textbf{Paper}(毕业论文),使论文使用正确的封面和独创性声明。

% \section{单面打印\& 双面打印}
% 学校并没有规定论文打印的方式,考虑到部分同学有双面打印的需求,可以在文档选项中使用oneside/twoside来切换单面打印和双面打印。

% \section{封面打印\& 装订}
% 建议大家去指定打印部门打印封面并装订,以免打印装订不合格。

% \section{批注}
% 在论文撰写过程中,pdf格式的论文,批注是一个问题,如果对\LaTeX 和基于Git的版本管理并不了解,就只能使用Adobe Acrobat、平板手写等软件,对pdf文件本身进行批注,相比于word确实有些麻烦。

% 强烈推荐使用Git\footnote{\url{https://git-scm.com/}}、Beyond Compare\footnote{\url{https://www.scootersoftware.com/}}等工具,辅以\LaTeX 本身的注释进行批注以及版本管理,非常清晰直观,操作也简单。

% \section{毕业设计与毕业论文的区别}
% 这里特别对使用本模板的本科同学们做出提醒,请查看毕业设计基本信息中的毕设类别,共有两类:\enquote{毕业设计}和\enquote{毕业论文}。因此在\verb!\documentclass[]{nwafuthesis}!的选项中需要标明\textbf{Design}(毕业设计)或者\textbf{Paper}(毕业论文),使论文使用正确的封面和独创性声明。

% \section{单面打印\& 双面打印}
% 学校并没有规定论文打印的方式,考虑到部分同学有双面打印的需求,可以在文档选项中使用oneside/twoside来切换单面打印和双面打印。

% \section{封面打印\& 装订}
% 建议大家去指定打印部门打印封面并装订,以免打印装订不合格。

% \section{批注}
% 在论文撰写过程中,pdf格式的论文,批注是一个问题,如果对\LaTeX 和基于Git的版本管理并不了解,就只能使用Adobe Acrobat、平板手写等软件,对pdf文件本身进行批注,相比于word确实有些麻烦。

% 强烈推荐使用Git\footnote{\url{https://git-scm.com/}}、Beyond Compare\footnote{\url{https://www.scootersoftware.com/}}等工具,辅以\LaTeX 本身的注释进行批注以及版本管理,非常清晰直观,操作也简单。

% \section{毕业设计与毕业论文的区别}
% 这里特别对使用本模板的本科同学们做出提醒,请查看毕业设计基本信息中的毕设类别,共有两类:\enquote{毕业设计}和\enquote{毕业论文}。因此在\verb!\documentclass[]{nwafuthesis}!的选项中需要标明\textbf{Design}(毕业设计)或者\textbf{Paper}(毕业论文),使论文使用正确的封面和独创性声明。

% \section{单面打印\& 双面打印}
% 学校并没有规定论文打印的方式,考虑到部分同学有双面打印的需求,可以在文档选项中使用oneside/twoside来切换单面打印和双面打印。

% \section{封面打印\& 装订}
% 建议大家去指定打印部门打印封面并装订,以免打印装订不合格。

% \section{批注}
% 在论文撰写过程中,pdf格式的论文,批注是一个问题,如果对\LaTeX 和基于Git的版本管理并不了解,就只能使用Adobe Acrobat、平板手写等软件,对pdf文件本身进行批注,相比于word确实有些麻烦。

% 强烈推荐使用Git\footnote{\url{https://git-scm.com/}}、Beyond Compare\footnote{\url{https://www.scootersoftware.com/}}等工具,辅以\LaTeX 本身的注释进行批注以及版本管理,非常清晰直观,操作也简单。

% \section{毕业设计与毕业论文的区别}
% 这里特别对使用本模板的本科同学们做出提醒,请查看毕业设计基本信息中的毕设类别,共有两类:\enquote{毕业设计}和\enquote{毕业论文}。因此在\verb!\documentclass[]{nwafuthesis}!的选项中需要标明\textbf{Design}(毕业设计)或者\textbf{Paper}(毕业论文),使论文使用正确的封面和独创性声明。

% \section{单面打印\& 双面打印}
% 学校并没有规定论文打印的方式,考虑到部分同学有双面打印的需求,可以在文档选项中使用oneside/twoside来切换单面打印和双面打印。

% \section{封面打印\& 装订}
% 建议大家去指定打印部门打印封面并装订,以免打印装订不合格。

% \section{附录的图表}

% 附录中的图表:

% \begin{figure}[htb]
%   \centering
%   \includegraphics[width=3cm]{nwafu-circle}
%   \caption{一个校徽}
% \end{figure}


% \begin{table}[htb]
%   \centering
%   \caption[城市人口]{城市人口数量排名 (source: Wikipedia)}
%   \begin{tabular}{lr}
%     \toprule
%     城市 & 人口 \\
%     \midrule
%     Mexico City & 20,116,842\\
%     Shanghai & 19,210,000\\
%     Peking & 15,796,450\\
%     Istanbul & 14,160,467\\
%     \bottomrule
%   \end{tabular}
% \end{table}

% \section{附录中的公式}

% 附录中的公式:

% \begin{align}
% d(\mathbf{p},\mathbf{q}) = d(\mathbf{q},\mathbf{p}) & = \sqrt{(q_1-p_1)^2 + (q_2-p_2)^2 + \cdots + (q_n-p_n)^2} \\
% & = \sqrt{\sum_{i=1}^n (q_i-p_i)^2}
% \end{align}
