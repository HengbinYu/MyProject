% 中文摘要
俄乌两国作为世界上重要的粮食大国,2022年的俄乌冲突对全球粮食安全造成了不小的冲击。其中2021年我国自乌克兰进口的玉米量约占我国总玉米进口量的30\%,在此背景下我国玉米等粮食的供应也受到了一定的影响。为了更好地保障我国在当前地缘冲突背景下的粮食安全,本文选择我国受到俄乌冲突影响较大的粮食-玉米作为研究对象,利用特征选择方法对各省玉米生产数据进行分析,从26个解释变量中选出了10个影响玉米产量的最重要因素,并对结果进行了分析与验证。

本文以模糊粗糙集属性约简算法作为核心研究方法,建立了分析玉米产量影响因素粮食生产模型,比较了基于Lasso的特征选择算法、基于随机森林变量重要性的特征选择算法(RF)与基于属性重要度的模糊粗糙集属性约简算法(FRAR)对于粮食生产数据在特征选择上的差异,发现FRAR方法在耗时上仅为Lasso方法的一半,略慢于RF方法。并采用随机森林预测模型数值实验对不同特征选择约简数据集进行分析与评价,得出FRAR约简数据集在训练耗时、训练集误差、测试集误差与预测精度方面较Lasso约简数据集有着很大的优势,且在实验结果上非常接近在各方面均表现最佳的RF约简数据集。最后对FRAR方法的特征选择结果分别从化肥投入、成本投入、排灌建设与机械化建设四个方面进行了可视化结果分析,得到了影响玉米产量的结论,并对我国粮食安全政策提供了科学建议。